\section{Diskussion}

% Zunächst ist die Area gleich!

In dieser Studie wurden vier CM1-Simulationen mit unterschiedlichem Mischungsverhältnis \(q_v\) bzw. CAPE durchgeführt, um die Entstehung und insbesondere die Präsenz von Hagel im bodennahen Modellniveau zu untersuchen. In den Simulationen entstehen mehrere Superzellen, wofür vor allem das verwendete Sounding von \textcite{weisman1982} im Zusammenspiel mit der Scherung des Windes verantwortlich ist. Allerdings werden die Gewittergebilde verbreiteter, je höher das CAPE ist. Gleiches gilt für den \(w \geq \SI{20}{\m\per\s}\) Aufwindbereich, welcher zusätzlich vor allem in der zweiten Simulationshälfte deutliche Größenunterschiede zwischen den Simulationen aufweist (vgl. \cref{fig:updraft} \textit{unten}). Die erhöhten Vertikalgeschwindigkeiten bei größerem CAPE sind dagegen schon kurz nach Simulationsstart nachzuweisen (vgl. \cref{fig:updraft} \textit{oben}).

Die zeitliche Entwicklung des Hagel-Mischungsverhältnisses \(q_h\) ähnelt konzeptionell eher dem Größenverlauf des Aufwindbereichs. Diese Tatsache legt nahe, dass die höhere Menge an bodennahem Hagel auf räumlich ausgeprägtere Superzellen mit insgesamt größerer Aufwindzone zurückzuführen ist und schnellere Vertikalwinde eine untergeordnete Rolle spielen. Insgesamt bleibt jedoch festzuhalten, dass für die CM1-Simulationen mit höherem CAPE über die gesamte Simulationsdauer und Domain gemittelt mehr Hagel am Boden nachweisbar ist. Eine quantitative Analyse einzelner Aufwindschlots gleicher Fläche der unterschiedlichen Simulationen könnte die Ursache(n) für die Zunahme von \(q_h\) weiter beleuchten.