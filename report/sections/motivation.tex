\section{Motivation}

% CAPE erklären
% updraft erklären
% Hagelproduktion in Gewitter anreißen und Relevanz herstellen
% Warum dieses Model?
% Meine Hypothese: Wie entsteht Hagel wo im Gewitter, warum Aufwind wichtig?

Superzellen sind bekannt für ihre Fähigkeit, enorme Mengen an Hagel zu produzieren, insbesondere große Hagelkörner mit einem Durchmesser von über \SI{4}{\cm}. Beispielsweise entstand das größte je in den USA gemessene Hagelkorn (mit einem Durchmesser von \SI{17.8}{\cm}) am 22. Juni 2003 in einer Superzelle \parencite{spektrum2003}, was ihre Gefährlichkeit verdeutlicht.

Diese Gewitterzellen zeichnen sich durch langlebige und hochreichende Mesozyklonen aus, in denen Aufwinde Geschwindigkeiten von über \SI{40}{\m\per\s} erreichen können \parencite{krider}. In diesen Aufwinden werden Wassertropfen in höhere, kältere Atmosphärenschichten transportiert, wo sie zu Eis gefrieren. Durch die Kollision mit unterkühlten Wassertropfen wächst das Hagelkorn weiter, was als Aggregation bezeichnet wird, bis es aufgrund des zunehmenden Einflusses der Schwerkraft absinkt und aus der Wolke fällt. Neben der Größe des Aufwindbereichs ist daher die maximale Vertikalgeschwindigkeit (und die entsprechende der Schwerkraft entgegengerichtete Autriebskraft) entscheidend für die Bildung großer Hagelmengen und letztere wird direkt von der Convective Available Potential Energy (CAPE) beeinflusst, auch bekannt als Labilitätsenergie. Der CAPE-Wert gibt an, wie viel Energie einem Luftpaket für den Aufstieg zur Verfügung steht und hängt unter anderem vom Feuchtegehalt der bodennahen Luftschicht ab.

Diese Analyse verändert daher indirekt das CAPE einer Atmosphäre, in welcher Superzellen simuliert werden. Anschließend wird die Vertikalgeschwindigkeit und horizontale Ausprägung der Aufwindzone quantifiziert und die Auswirkungen auf den bodennahen Hagel betrachtet.

% die Aufwinde sind viel schneller in meinen Simulationen als in echt!	